Prior work on 3D Face Reconstruction is substantially large. To make the analysis easier, we classify the works on the baiss of the general 3D reconstruction approach they follow, namely : SfM based Multi-view Reconstruction, Photometric Stereo, Single Image 3D Face Reconstruction

\section{SfM based Multi-view Reconstruction} A lot of multi-view reconstruction methods employ a Structure-from-Motion pipeline ~\cite{gotardo2015photogeometric, lin2010accurate, fidaleo2007model} but with unconvincing results on unconstrained in-the-wild videos ~\cite{hernandez2017accurate}. ~\cite{brand2001morphable} and ~\cite{shi2014automatic} use 3D Morphable Model~\cite{blanz1999morphable} for fitting shapes on every frame after computing correspondences among them. This restricts the reconstruction to a low-dimensional linear subspace. The current state-of-the-art approach by Hernandez ~\etal ~\cite{hernandez2017accurate} uses 3DMM as a prior instead to search for correspondences among frames. This allowed them to achieve state-of-the-art results in unconstrained multi-view face reconstruction. However their method requires camera intrinsics to be known and the output is still constrained to a linear basis. We use this method as one of the baselines for comparison.

\section{Photometric Stereo} Photometric stereo based methods have proven effective for large unconstrained collection of photos ~\cite{kemelmacher2011face, kemelmacher2013internet, roth2015unconstrained}. ~\cite{kemelmacher2011face} generates a 2.5D face surface by using SVD to find the low rank spherical harmonics. Roth \etal ~\cite{roth2015unconstrained} expand on it to handle pose variations and the scale ambiguity prevalent in the former method. They further expand their work in ~\cite{roth2016adaptive} where they fit a 3DMM to 2D landmarks for every image and optimize for the lighting parameters rather than SVD based factorization. Suwajanakorn ~\etal ~\cite{suwajanakorn2014total} use shape from shading coupled with 3D flow estimation to target uncalibrated video sequences. While these methods capture fine facial features, most of them rely on simplified lighting, illumination and reflectance models, resulting in specularities and unwanted facial features showing up on the mesh.

\section{Single Image 3D Face Reconstruction} 3D Morphable Models have successfully been used as prior for modeling faces from a single image ~\cite{blanz1999morphable, breuer2008automatic, zhu2015high, saito2017photorealistic, jiang20183d, richardson20163d, tuan2017regressing}. Facial landmarks have commonly been used in conjunction with 3DMMs for the reconstruction ~\cite{zhu2015high, aldrian2010linear, kemelmacher20113d, dou2014robust}. While landmarks are informative for 3D reconstruction, relying primarily on them results in generic looking meshes which lack recognizable detail. More recently, convolutional neural networks have been put to use for directly regressing the parameters of the 3D Morphable Model ~\cite{zhu2016face, jourabloo2016large}. To overcome the limited expressivity of 3DMMs, recent methods have tried to reconstruct unrestricted geometry, by predicting  a volumetric representation ~\cite{jackson2017large}, UV map \cite{feng2018joint}, or depth map \cite{sela2017unrestricted}. However, the underlying training data of these methods has been limited to synthetic data generated using 3DMMs or course meshes fit using landmarks. Thus the ability of these methods to generalize to `in-the-wild' images and face geometry is still quite limited. While single image reconstruction is of great research interest, we believe multi-view consistency is crucial for generating accurate 3D face representations, specially given the limited data available for faces.  For a more comprehensive literature review of monocular 3D Face Reconstruction, we direct the readers to ~\cite{zollhofer2018state}.
%-------------------------------------------------------------------------

